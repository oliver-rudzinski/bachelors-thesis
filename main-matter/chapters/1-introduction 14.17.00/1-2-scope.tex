The goal of the superordinate project of this bachelor's thesis is to design and implement a visual, interactive DataOps use case. Specifically, a fictional, idealized retail business use case has been ideated and is now subject to implementation. The use case idea works with a number of retail branches of a retail company that sells different goods to its customers. Based on the purchases, the \ac{pos} data (i.e., receipts) can be used for analytics and \ac{bi} purposes. Specifically, these data can be leveraged for \ac{mba}. The use case at hand remains entirely fabricated, allowing for an isolated, non-business-critical proof of concept design environment. Apart from the pure data analytics part, the desired solution includes a data generator of pseudoreal input data as well as a web \acs{ui} for visualization and presentation purposes. These aspects lie outside the scope of this thesis' project. It focusses on the analytics-driven area of the use case. Specifically, the analytics solution of the demonstration use case is a data pipeline which receives input data and performs individual steps in order to generate an \ac{mba}, which results could then be visualized inside the \ac{ui}.