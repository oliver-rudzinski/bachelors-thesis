%===========================================================================
%	I. Introduction
%===========================================================================

Data analytics has become a top-priority discipline for organizations of all industries. It is the crucial driver for several business use cases, including \ac{dwh} optimization, forecasting, customer and social analysis as well as fraud detection \cite{Statista}. Additionally, it is expected to aid the overall business decision-making process \cite{Souibgui2019}. Unfortunately, the development of such data analytics solutions remains complicated, and often, unsuccessful \cite{DataKitchen2019}.

This can be seen 87 percent of data science projects never reaching the state of production-grade solutions \cite{VentureBeat}. Moreover, Gartner reports that projects driven by \ac{ai} can mostly not be industrialized, resulting in bad scalability within the organizations \cite{White2019}. One reason for this might be that data analytics projects are still conducted in a non-dynamic and sequential way.

\textit{DataOps} is a new working model for conducting and maintaining data analytics projects, combining findings and best practices from manufacturing processes, \textit{DevOps} software development and \textit{agile} project management \cite[17\psqq]{Bergh2019}. Since data analytics solutions are data-driven software solutions, testing is an important factor of the entire DataOps development process \cite[40\psqq]{Bergh2019}. Testing is required to build up confidence and trust in the solution. This should not only prove that the solution is functional but that is also provides valuable and correct outcomes.

This bachelor's thesis deals with the topic of DataOps testing inside an exemplarily chosen Big Data \ac{bi} retail use case, specifically the \acf{mba}. Even though DataOps paves the path for testing by providing high-level testing ideologies, it does not specifically provide an actual testing framework. The project behind this thesis designs and implements a prototype DataOps testing framework and evaluates it within the given use case. The research goal of this thesis is to understand the area and process of DataOps testing, to find testing similarities and differences with DevOps, as well as to find technical limitations of DataOps testing.

\section*{Relation to Project Environment} \addtocounter{section}{1}
	The project is conducted within the \textit{Analytics} department of \textit{DXC Technology Company}. The Chief Technology Office inside this department has received the thought ownership regarding DataOps within DXC Technology. This division is currently implementing DataOps as a competence inside its service portfolio and is working on DataOps realization projects with several clients of different business areas. A proof of concept, outlining the features and advantages of DataOps, is desired. On the one hand, it could help get potential customers interested in DataOps and DXC realizing it. On the other hand, such a project could also be used in existing client workshops and for employee education purposes. The latter could also improve the performance of current and future DataOps projects conducted by DXC Analytics.
	\label{sec:1-relation}
	
\section*{Project Scope} \addtocounter{section}{1}
	The goal of the superordinate project of this bachelor's thesis is to design and implement a visual, interactive DataOps use case. Specifically, a fictional, idealized retail business use case has been ideated and is now subject to implementation. The use case idea works with a number of branches within a retail company that sells different goods to its customers. Based on the purchases, the \ac{pos} data (i.e., receipts) can be used for analytics and \ac{bi} purposes. Specifically, these data can be leveraged for \ac{mba}. The use case at hand remains entirely fabricated, allowing for an isolated, non-business-critical proof-of-concept design environment. Apart from the pure data analytics part, the desired solution includes a data generator for idealized \ac{pos} input data as well as a web \acs{ui} for visualization and presentation purposes. These aspects lie outside the scope of this thesis' project. It rather focusses on the analytics-driven area, or back end, of the use case. Specifically, the analytics solution of the demonstration use case is a data pipeline which receives input data and performs individual steps in order to generate an \ac{mba} report file. This data pipeline is subject to be enhanced with DataOps testing capabilities.
	\label{sec:1-scope}
	
\section*{Task Definition} \addtocounter{section}{1}
	The previously described use case analytics solution has already been developed. Now, the solution at needs to be evaluated from a DataOps perspective, redesigning it to comply with state-of-the-art DataOps methodologies and standards. Then, the new solution needs to be enhanced with suitable testing frameworks, considering the use case circumstances and priorities. All required infrastructure for reaching the project targets needs to be realized and deployed within \ac{aws}, the Amazon cloud platform. Additionally, for the sake of scalability and process isolation, all analytical processes are desired to be designed in a server-less approach.
	\label{sec:1-task}
	
\section*{Chapter Overview} \addtocounter{section}{1}
	This chapter, the \nameref{chap:introduction} chapter, presented the high-level issues of traditional analytics solution development and proposed DataOps as an enhancement and described the practical tasks that are to be performed.

Chapter \ref{chap:theoretical-backgrounds}, the \nameref{chap:theoretical-backgrounds} chapter, introduces DataOps and its key principles and methodologies as well as general testing frameworks for both software and data quality testing.

Chapter \ref{chap:actual-state-analysis}, the \nameref{chap:actual-state-analysis} chapter, describes the preexisting analytics solution for the given use case and states present issues which stand in contrast to the task definition.

Chapter \ref{chap:testing-framework}, the \nameref{chap:testing-framework} chapter, designs the method for implementing holistic DataOps testing inside the use case. It considers the findings from Chapters \ref{chap:theoretical-backgrounds} and \ref{chap:actual-state-analysis}.

Chapter \ref{chap:implementation}, the \nameref{chap:implementation} chapter, enables DataOps methodologies described in Chapter \ref{chap:theoretical-backgrounds} and implements the testing framework from Chapter \ref{chap:testing-framework}, taking missing aspects required by the project target definition into account.

Chapter \ref{chap:solution-evaluation}, the \nameref{chap:solution-evaluation} chapter, evaluates the new solution based on the general workflow of DataOps Testing, its relation to DevOps Testing as well as possible limitations.

Chapter \ref{chap:conclusion}, the final \nameref{chap:conclusion} chapter, summarizes all findings, proposes further research and enhancement, and concludes this thesis.

	\label{sec:1-chapters}